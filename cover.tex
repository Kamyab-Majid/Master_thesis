
\degree{\MSc}

\dept{Chemical Engineering}

\field{Chemical Engineering}

%\field{ }



\permanentaddress{12-203 Donadeo Innovation Centre for Engineering \\
	University of Alberta \\
	Edmonton, Alberta \\
	Canada, T6G 1H9}
\examiners{Youssef Belhamadia, Stevan Dubljevic, Richard Thompson}%

\convocationseason{Majid Kamyab,}


\title{Low-Level control of small scale helicopter using Soft Actor-Critic method}
\author{Majid Kamyab}%
\admin
\doublespacing        %abstract has to be double-spaced




\begin{abstract}
\truedoublespacing

Unmanned Aerial Vehicles (UAVs), or drones, have been employed in a variety of applications, ranging from surveillance to emergency operations. These systems comprise an "inner loop" that provides stability and control and an "outer loop" in charge of mission-level tasks, such as way-point navigation. Despite their inherent instability, different techniques for controlling these robots have been devised under stable environmental conditions. However, these algorithms must know a robot's dynamics to be effective; furthermore, more complex control is necessary for UAVs to perform in unstable environmental conditions. In this research, a simulated drone has been successfully controlled using model-free reinforcement learning with no prior knowledge of the robot's model. Soft Actor-Critic (SAC) method is trained to perform low-level control of a small-scaled helicopter in a set-point control system. First, a simulation environment is created in which all tests were carried out and then it is shown that SAC can not only develop a strong policy, but it can also deal with unknown circumstances. The result obtained by the SAC agent is also compared to a sliding mode controller to compare the capability of this method to a traditional nonlinear control method. The SMC method proved to be superior by a steady state error of 0, compared to a steady state error of 0.05\% for the SAC agent. However, the SAC agent is a model free technique which does not have access to the model of the helicopter, on the other hand the SMC is a model based technique whuch needs the system identification of the helicopter system.
\end{abstract}



%\begin{preface}
%to be added ???


%\end{preface}




\begin{acknowledgements}
I want to express my gratitude to my distinguished supervisor, Dr. Dubljevic, for his essential supervision, support, and instruction throughout my Master's degree. My thanks go to the Faculty of Engineering for providing me with the money to pursue my studies at the University of Alberta's Department of Chemical Engineering. In addition, I'd like to thank Dr. Koch for his invaluable assistance. I'd want to thank my friends, lab mates, colleagues, and research team – Hamid Khatibi – for a memorable time spent together in academic and social contexts. My thanks also go to my mother, brother, and father for their support during my education.

\end{acknowledgements}





